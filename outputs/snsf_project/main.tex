% SNSF Project Funding - Research Plan Template
% Based on successful SNSF project proposals
% Maximum: 15 pages / 60,000 characters (17 pages / 68,000 for collaborative)

\documentclass[english, 11pt]{article}

%% ============================================================
%% PACKAGES
%% ============================================================
\usepackage{amssymb}
\usepackage{amsfonts}
\usepackage{amsmath}
\usepackage{bm}
\usepackage[T1]{fontenc}
\usepackage[utf8]{inputenc}
\usepackage[a4paper, verbose, tmargin=1.9cm, bmargin=1.9cm, lmargin=2.5cm, rmargin=2.5cm]{geometry}
\usepackage{float}
\usepackage{setspace}
\usepackage[round]{natbib}
\usepackage{caption}
\usepackage{babel}
\usepackage{lscape}
\usepackage{mathptmx}  % Times font
\PassOptionsToPackage{hyphens}{url}
\usepackage{tocloft}
\usepackage{xcolor}
\usepackage{url}
\usepackage{hyperref}
\hypersetup{
    colorlinks=true,
    linkcolor=black,
    filecolor=black,
    urlcolor=teal,
    citecolor=teal,
}
\usepackage{graphicx}
\usepackage[labelfont=sc]{subcaption}
\usepackage[bottom, hang]{footmisc}

%% ============================================================
%% DOCUMENT SETTINGS
%% ============================================================
\setlength\parindent{0pt}
\onehalfspacing  % 1.5 line spacing (SNSF requirement)

% Rename Bibliography section
\addto{\captionsenglish}{\renewcommand{\refname}{Bibliography}}

% Prevent orphans and widows
\clubpenalty=10000
\widowpenalty=10000
\displaywidowpenalty=10000

% Footnote formatting
\setlength{\footnotemargin}{0em}

% Blank footnote command (for author affiliations)
\makeatletter
\def\blfootnote{\xdef\@thefnmark{}\@footnotetext}
\makeatother

%% ============================================================
%% DOCUMENT START
%% ============================================================
\begin{document}

%% ============================================================
%% TITLE PAGE
%% ============================================================
\begin{singlespace}
\begin{center}

\LARGE{[PROJECT TITLE]}

\bigskip

\Large{Research proposal for a project grant of the\\
Swiss National Science Foundation}

\bigskip
\bigskip

\begin{tabular}{c}
{\Large [Applicant Name]$^{*}$ (PI)} \\
% Uncomment for collaborative projects:
% {\Large [Co-Applicant Name]$^{**}$ (Project Partner)} \\
\\
\large [Month Year]
\end{tabular}
\end{center}
\end{singlespace}

% Author affiliations as footnotes
\blfootnote{$^{*}$[Institution], [Department], [Address], [Country]; email: [email@example.com].}
% Uncomment for additional authors:
% \blfootnote{$^{**}$[Institution], [Department], [Address], [Country]; email: [email@example.com].}

\vspace{1.5cm}

\setcounter{tocdepth}{2}
\tableofcontents

\pagebreak

%% ============================================================
%% SECTION 1: SUMMARY
%% ============================================================
\section{Summary of the Research Plan}

[Write a standalone summary (~1 page) covering:
- Project title and core research question(s)
- Methodology overview
- Expected outcomes and impact
- Why this research matters

TIP: Write this section LAST after completing all other sections.]

\pagebreak

%% ============================================================
%% SECTION 2: RESEARCH PLAN
%% ============================================================
\section{Research Plan}

%% ------------------------------------------------------------
%% 2.1 STATE OF RESEARCH IN THE FIELD
%% ------------------------------------------------------------
\subsection{Current state of research in the field}

[Provide context for your research (2-3 pages):
- Introduce the general field and its importance
- Summarize key findings and established approaches
- Identify research gaps and limitations
- Position your project within the landscape

Include subsections for different themes if needed:]

\vspace{0.3cm}
\noindent\textit{[Theme 1: Title]}

[Discussion of first major theme in the literature...]

\vspace{0.3cm}
\noindent\textit{[Theme 2: Title]}

[Discussion of second major theme...]

\vspace{0.3cm}
\noindent\textit{[Theme 3: Title]}

[Discussion of third major theme...]


%% ------------------------------------------------------------
%% 2.2 STATE OF OWN RESEARCH
%% ------------------------------------------------------------
\subsection{Current state of own research}

[Demonstrate your capability to execute this project (1-2 pages)]

\vspace{0.3cm}
\noindent\textit{Most closely related previous work}

[Describe your prior work that is most relevant to this proposal. Include:
- Key publications and their contributions
- Methods and expertise you've developed
- How this experience enables the proposed research]

\vspace{0.3cm}
\noindent\textit{Specific preparatory work}

[Describe preparatory work done for this proposal:
- Pilot studies or preliminary data
- Technical infrastructure development
- Data collection and evaluation
- Partnerships established]


%% ------------------------------------------------------------
%% 2.3 DETAILED RESEARCH PLAN
%% ------------------------------------------------------------
\pagebreak
\subsection{Detailed research plan}

[Core of the proposal (6-8 pages). Include all subsections below.]

\subsubsection*{General objectives of the proposed research plan}

[State the primary goals and what you aim to achieve. Be specific about:
- What questions you will answer
- What contributions you will make
- What you explicitly will NOT address (scope boundaries)]


\subsubsection*{Research team and responsibilities}

[Describe the team composition and roles:
- Principal Investigator responsibilities
- Project partners (if collaborative)
- Research assistants and their tasks
- External collaborators or consultants]


\subsubsection*{Challenges for empirical research}

[Identify key empirical challenges:
- Measurement challenges
- Identification challenges
- Data access challenges
- Technical challenges

Explain how your approach addresses these challenges.]


\subsubsection*{Methods: [Your methodological approach]}

[Detailed methodology section. Structure depends on your field but should include:
- Conceptual framework
- Data sources and collection
- Analytical methods with justification
- Experimental design (if applicable)

Include figures to illustrate key concepts:]

\vspace{0.3cm}

% Example figure
% \begin{figure}[htb]
% \footnotesize
% \begin{center}
% \caption{[Figure Title]}
% \label{fig:methodology}
% \includegraphics[width=0.7\textwidth]{fig/methodology.pdf}
% \end{center}
% \begin{singlespace}
% \footnotesize{\textit{Notes:} [Figure description and explanation.]}
% \end{singlespace}
% \end{figure}


\pagebreak
\subsubsection*{[Experiment/Study 1: Title]}

\vspace{0.3cm}
\noindent\textit{Context and experimental design}

[Describe the first major component of your research:
- Research question addressed
- Hypotheses
- Experimental/study design
- Expected sample sizes]

\vspace{0.3cm}
\noindent\textit{Input data and configuration}

[Describe data requirements:
- Data sources
- Data collection procedures
- Sample selection criteria
- Quality assurance measures]

\vspace{0.3cm}
\noindent\textit{Data analysis plan}

[Describe analytical approach:
- Statistical methods
- Robustness checks
- How results will be interpreted]


% Repeat structure for additional experiments/studies:
% \pagebreak
% \subsubsection*{[Experiment/Study 2: Title]}
% ...


\subsubsection*{Risk assessment and mitigation}

[Identify major risks and mitigation strategies:]

\begin{table}[h]
\centering
\small
\begin{tabular}{|p{3cm}|p{2cm}|p{2cm}|p{4cm}|p{3cm}|}
\hline
\textbf{Risk} & \textbf{Probability} & \textbf{Impact} & \textbf{Mitigation} & \textbf{Fallback} \\
\hline
[Risk 1] & Medium & High & [Strategy] & [Alternative approach] \\
\hline
[Risk 2] & Low & High & [Strategy] & [Alternative approach] \\
\hline
[Risk 3] & Medium & Medium & [Strategy] & [Alternative approach] \\
\hline
\end{tabular}
\end{table}


%% ------------------------------------------------------------
%% 2.4 SCHEDULE AND MILESTONES
%% ------------------------------------------------------------
\subsection{Schedule and milestones}

[Provide timeline (~1 page) with:
- Gantt chart or timeline table
- Key milestones
- Deliverables
- Go/no-go decision points]

% Include schedule figure
% \begin{figure}[h!]
% \caption{Schedule and milestones}
% \begin{center}
% \includegraphics[width=\textwidth]{fig/schedule.pdf}
% \end{center}
% \begin{singlespace}
% \footnotesize{\textit{Notes:} [Explanation of the schedule. Indicate which tasks are led by PI vs research assistants vs developers. Mark major milestones.]}
% \end{singlespace}
% \end{figure}


%% ------------------------------------------------------------
%% 2.5 RELEVANCE AND IMPACT
%% ------------------------------------------------------------
\subsection{Relevance and impact}

[Articulate significance (1-2 pages):
- Scientific contribution to the field
- Broader societal impact
- Policy implications (if applicable)
- Methodological contributions
- Training and educational value]

\vspace{0.3cm}
\noindent\textit{Dissemination strategy}

[Describe how results will be shared:
- Conference presentations
- Working papers
- Open access policy
- Data and code sharing]

\vspace{0.3cm}
\noindent\textit{Publication strategy}

[Target outlets for publications:
- Primary target journals
- Alternative venues
- Timeline for submissions]


%% ============================================================
%% BIBLIOGRAPHY
%% ============================================================
\clearpage
\bibliographystyle{apalike}
\bibliography{ref}

\end{document}
